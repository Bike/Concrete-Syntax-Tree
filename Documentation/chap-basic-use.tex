\chapter{Basic use}

In this chapter, we describe the basic functionality that \sysname{}
provides for manipulating concrete syntax trees.

\Defclass {cst}

This class is the base class for all concrete syntax trees.

\Definitarg {:parent}

This initialization argument is accepted by all subclasses of concrete
syntax trees.  If the syntax tree is a top-level syntax tree, then
\texttt{nil} should be provided as the value for this initialization
argument.  Otherwise, the value should be the concrete syntax tree
that is the parent of this one.

\Definitarg {:source}

This initialization argument is accepted by all subclasses of concrete
syntax trees.  The value of this initialization argument is a
client-specific object that indicates the origin of the source code
represented by this concrete syntax tree.  A value of \texttt{nil}
indicates that the origin of the source code represented by this concrete
syntax tree is unknown.  The default value (if this initialization
argument is not provided) is \texttt{nil}.

\Defgeneric {parent} {cst}

This generic function returns the parent of \texttt{cst} as provide by
the initialization argument \texttt{:parent} when \texttt{cst} was
created.

\Defgeneric {source} {cst}

This generic function returns the origin information of \texttt{cst}
as provide by the initialization argument \texttt{:source} when
\texttt{cst} was created.

\Defclass {expression-cst}

This class is a subclass of the class \texttt{cst}.  It is used for
concrete syntax trees that represent \commonlisp{} expressions.

\Definitarg {:raw}

The value of this initialization argument is the raw \commonlisp{}
expression that this concrete syntax tree represents.

\Defclass {cons-cst}

This class is a subclass of the class \texttt{expression-cst}.
