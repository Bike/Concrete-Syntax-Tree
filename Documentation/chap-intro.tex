\chapter{Introduction}
\pagenumbering{arabic}%

\section{Purpose}

\sysname{} is a library for manipulating \commonlisp{} source code
enhanced with information about the origin of the source.  It is
intended to solve a general problem that occurs in particular in
\commonlisp{} file compilers.  If a file compiler compiles the file by
first calling the \commonlisp{} \texttt{read} function on every
top-level expression in the file, then the information about the
origin of those expressions is lost.  This loss of information has a
serious negative impact on compiler messages, because the application
programmer does not get direct information about the origin of the
code that the compiler message refers to.  The solution to this
problem involves what is called \emph{source tracking}, which
basically means that we need to keep track of this origin information
from the initial file, throughout the compilation process, all the way
to the executable code.

One requirement for improved source tracking is that the source code
must be read by an improved version of the \texttt{read} function.
A typical solution would be to make \texttt{read} keep a hash table
that associates the expressions being read to the location in the
file.  But this solution only works for freshly allocated
\commonlisp{} objects.  It will not work for code elements such as
numbers, characters, or symbols, simply because there may be several
occurrences of similar code elements in the source.

The solution provided by this library is to manipulate \commonlisp{}
source code in the form of a \emph{concrete syntax tree}, or CST for
short.  A concrete syntax tree is simply a wrapper (in the form of a
standard instance) around a \commonlisp{} expression, providing the
additional information required for source tracking.  In order to make
the manipulation of concrete syntax trees as painless as possible for
client code, this library provides a set of functions that mimic the
ones that would be used on raw \commonlisp{} code, such as
\texttt{first}, \texttt{rest}, \texttt{consp}, \texttt{null}, etc.

Since the exact nature of the origin information in a CST depends on
the Common Lisp implementation and the purpose of wanting to track
that origin, we do not impose a particular structure of this
information.  Instead, we just propagate this information as much as
possible through the functions in this library that manipulate the
source code in the form of CSTs.

For example, we provide code utilities for canonicalizing
declarations, parsing lambda lists, separating declarations and
documentation strings and code bodies, checking whether a form is a
proper list, etc.  All these utilities manipulate the code in the form
of a CST, and provide CSTs as a result of the manipulation that
propagates the origin information as much as possible.

In particular, we provide an "intelligent macroexpander".  This
function takes an original CST and the result of macroexpanding the
RAW code version of that CST, and returns a new CST representing the
expanded code in such a way that as much as possible of the origin
information is preserved.
