\chapter{Parsing lambda lists}
\label{chap-user-parsing-lambda-lists}

The \sysname{} library contains a framework for parsing lambda lists.
This framework contains functions for parsing the types of lambda
lists that specified in the \commonlisp{} standard, but it also
contains a protocol that allows different implementations to specify
additional lambda-list keywords, and to specify how lambda lists
containing these additional keywords should be parsed.

\section{Classes for grammar symbols}

\Defclass {grammar-symbol}

This class is the root of all grammar-symbol classes.

\subsection{Classes for parameter groups}
\label{sec-parsing-lambda-lists-parameter-groups}

\Defclass {parameter-group}

This class is the root class of all classes that represent parameter
groups.  It is a subclass of the class \texttt{grammar-symbol}.

\Definitarg {:children}

This initialization argument can be used with all subclasses of the
class named \texttt{parameter-group}.  For parameter groups that have
no lambda-list keywords, such as all the parameter groups
corresponding to required parameters, the value of the argument is a
(possibly empty) list of parameters.  For parameter groups that have
associated lambda-list keywords, the value of the argument includes
those lambda-list keywords in addition to the parameters themselves.

\Defclass {singleton-parameter-group-mixin}

This class is used as a superclass of all classes representing
parameter groups with a keyword followed by a single parameter.

\Definitarg {:parameter}

This initialization argument can be used with subclasses of the class
named \texttt{singleton-parameter-group-mixin}, but the parser does
not use it.  Instead, it passes the \texttt{:children} initialization
argument.  An appropriate \texttt{:after} method on
\texttt{initialize-instance} splits the children into the lambda-list
keyword itself and the single following parameter.

\Defgeneric {parameter} {singleton-parameter-group-mixin}

This generic function returns a concrete syntax tree representing the
single parameter of its argument.

\Defclass {multi-parameter-group-mixin}

This class is used as a superclass of all classes representing
parameter groups with or without a keyword followed by a (possibly
empty) list of parameters.

\Definitarg {:parameters}

This initialization argument can be used with subclasses of the class
named \texttt{multi-parameter-group-mixin}, but the parser does
not use it.  Instead, it passes the \texttt{:children} initialization
argument.  An appropriate \texttt{:after} method on
\texttt{initialize-instance} computes this initialization argument
from the children.

\Defgeneric {parameters} {multi-parameter-group-mixin}

This generic function returns a list of concrete syntax trees
representing the parameters of its argument.

\Defclass {implicit-parameter-group}

This class is the root class of all classes that represent parameter
groups that are \emph{not} introduced by a lambda-list keyword, which
is all the different classes representing required parameter groups.

This class is a subclass of the class named \texttt{parameter-group}
and the class named \texttt{multi-parameter-group-mixin}.

\Defclass {explicit-parameter-group}

This class is the root class of all classes that represent parameter
groups that \emph{are} introduced by a lambda-list keyword.

This class is a subclass of the class \texttt{parameter-group}.

\Definitarg {:keyword}

This initialization argument can be used with subclasses of the class
named \texttt{explicit-parameter-group}, but the parser does not use
it.  Instead, it passes the \texttt{:children} initialization
argument.  An appropriate \texttt{:after} method on
\texttt{initialize-instance} computes this initialization argument
from the children.

\Defgeneric {keyword} {explicit-parameter-group}

This generic function returns the lambda-list keyword of its argument.

\Defclass {explicit-multi-parameter-group}

This class is the root class of all classes that represent parameter
groups that are introduced by a lambda-list keyword, and that can take
an arbitrary number of parameters.

This class is a subclass of the class named
\texttt{explicit-parameter-group} and of the class named
\texttt{multi-parameter-group-mixin}

\Defclass {ordinary-required-parameter-group}

This class represents the list of required parameters in all lambda
lists that only allow a simple variables to represent a required
parameter.

This class is a subclass of the class \texttt{implicit-parameter-group}.

\Defclass {optional-parameter-group}

This class is the root class of all classes representing optional
parameter groups.

It is a subclass of the class \texttt{explicit-multi-parameter-group}.

\Defclass {ordinary-optional-parameter-group}

This class represents the list of optional parameters in all lambda
lists that allow for an optional parameter to have a form representing
the default value and a \texttt{supplied-p} parameter.

This class is a subclass of the class \texttt{optional-parameter-group}.

\Defclass {key-parameter-group}

This class is the root class of all parameter groups that are
introduced by the lambda-list keyword \texttt{\&key}.

This class is a subclass of the class \texttt{explicit-multi-parameter-group}.

\Defgeneric {allow-other-keys} {key-parameter-group}

This function can be called on any instance of the class
\texttt{key-parameter-group}.  If the lambda-list keyword
\texttt{\&allow-other-keys} is present in this key parameter group,
then \texttt{allow-other-keys} returns a concrete syntax
tree for that lambda-list keyword.  If the lambda-list keyword
\texttt{\&allow-other-keys} is \emph{not} present, this function
returns \texttt{nil}.

\Defclass {ordinary-key-parameter-group}

This class represents the list of \texttt{\&key} parameters in all
lambda lists that allow for the parameter to have a form representing
the default value and a \texttt{supplied-p} parameter.

This class is a subclass of the class \texttt{key-parameter-group}.

\Defclass {generic-function-key-parameter-group}

This class represents the list of \texttt{\&key} parameters in a
generic-function lambda list.  This type of lambda list only allows
for the parameter to have a variable name and a keyword.

This class is a subclass of the class \texttt{key-parameter-group}.

\Defclass {aux-parameter-group}

This class represent the list of \texttt{\&aux} parameters in all
lambda lists that allow for such parameters.

This class is a subclass of the class \texttt{explicit-multi-parameter-group}.

\Defclass {generic-function-optional-parameter-group}

This class represents the list of optional parameters in a
generic-function lambda list.  This type of lambda list only allows
for the parameter to have a variable name.

This class is a subclass of the class \texttt{optional-parameter-group}.

\Defclass {specialized-required-parameter-group}

This class represents the list of required parameters in a specialized
lambda list, i.e., the type of lambda list that can be present in a
\texttt{defmethod} definition.  In this type of lambda list, a
required parameter may optionally have a \emph{specializer} associated
with it.

This class is a subclass of the class \texttt{implicit-parameter-group}.

\Defclass {destructuring-required-parameter-group}

This class represents the list of required parameters in a destructuring
lambda list, i.e. the kind of lambda list that can be present in a
\texttt{defmacro} definition, both as the top-level lambda list and as
a nested lambda list where this is allowed.

This class is a subclass of the class \texttt{implicit-parameter-group}.

\Defclass {singleton-parameter-group}

This class is the root class of all parameter groups that consist of a
lambda-list keyword followed by a single parameter.  It is a subclass
of the class \texttt{explicit-parameter-group}.

This class is a subclass of the class named
\texttt{explicit-parameter-group} and of the class named \texttt{
  singleton-parameter-group-mixin}.

\Defclass {ordinary-rest-parameter-group}

This class represents parameter groups that have either the
lambda-list keyword \texttt{\&rest} or the lambda-list keyword
\texttt{\&body} followed by a simple-variable.

This class is a subclass of the class \texttt{singleton-parameter-group}.

\Defclass {destructuring-rest-parameter-group}

This class represents parameter groups that have either the
lambda-list keyword \texttt{\&rest} or the lambda-list keyword
\texttt{\&body} followed by a destructuring parameter.

This class is a subclass of the class \texttt{singleton-parameter-group}.

\Defclass {environment-parameter-group}

This class represents parameter groups that have the lambda-list
keyword \texttt{\&environment} followed by a simple-variable.

This class is a subclass of the class \texttt{singleton-parameter-group}.

\Defclass {whole-parameter-group}

This class represents parameter groups that have the lambda-list
keyword \texttt{\&whole} followed by a simple-variable.

This class is a subclass of the class \texttt{singleton-parameter-group}.

\subsection{Classes for individual parameters}

\Defgeneric {name} {parameter}

\Defclass {parameter}

This class is the root class of all classes that represent individual
lambda-list parameters.

This class is a subclass of the class \texttt{grammar-symbol}.

\Defclass {form-mixin}

This mixin class is a superclass of subclasses of \texttt{parameter}
that allow for an optional form to be supplied, which will be
evaluated to supply a value for a parameter that is not explicitly
passed as an argument.

\Definitarg {:form}

This initialization argument can be used when an instance of
(a subclass of) the class \texttt{form-mixin} is created.  The value
of this initialization argument is either a concrete syntax tree
representing the form that was supplied in the lambda list, or
\texttt{nil}, indicating that no form was supplied.

\Defgeneric {form} {form-mixin}

This generic function can be applied to instances of all subclasses of
the class \texttt{form-mixin}.  It returns the value that was supplied
using the \texttt{:form} initialization argument when the instance was
created.

\Defclass {supplied-p-mixin}

This mixin class is a superclass of subclasses of \texttt{parameter}
that allow for an optional supplied-p parameter to be supplied.  At
run-time, the value of this parameter indicates whether an explicit
argument was supplied that provides a value for the parameter.

\Definitarg {:supplied-p}

This initialization argument can be used when an instance of (a
subclass of) the class \texttt{supplied-p-mixin} is created.  The
value of this initialization argument is either a concrete syntax tree
representing the supplied-p parameter that was given in the lambda
list, or \texttt{nil}, indicating that no supplied-p parameter was
given.

\Defgeneric {supplied-p} {supplied-p-mixin}

This generic function can be applied to instances of all subclasses of
the class \texttt{supplied-p-mixin}.  It returns the value that was provided
using the \texttt{:supplied-p} initialization argument when the instance was
created.

\Defclass {simple-variable}

This class represents lambda-list parameters that must be simple
variables, for example the required parameters in an ordinary lambda
list, or the parameter following the \texttt{\&environment}
lambda-list keyword.

This class is a subclass of the class \texttt{parameter}.

\Defclass {ordinary-optional-parameter}

This class represents an optional parameter of the form that is
allowed in an ordinary lambda list, i.e., a parameter that, in
addition to the parameter name, can have a form representing the
default value and an associated \texttt{supplied-p} parameter.

This class is a subclass of the classes
\texttt{parameter} and \texttt{form-mixin}.

\Defclass {ordinary-key-parameter}

This class represents a \texttt{\&key} parameter of the form that is
allowed in an ordinary lambda list, i.e., a parameter that, in
addition to the parameter name, can have a keyword, a form
representing the default value, and an associated \texttt{supplied-p}
parameter.

This class is a subclass of the classes
\texttt{parameter} and \texttt{form-mixin}.

\Defclass {generic-function-key-parameter}

This class represents a \texttt{\&key} parameter of the form that is
allowed in a generic-function lambda list, i.e., a parameter that, in
addition to the parameter name, can have a keyword, but no form
representing the default value, and no associated \texttt{supplied-p}
parameter.

This class is a subclass of the class \texttt{parameter}.

\Defclass {aux-parameter}

This class represents an \texttt{\&aux} parameter, i.e., a parameter
that, in addition to the parameter name, can have a form representing
the default value.

This class is a subclass of the classes
\texttt{parameter} and \texttt{form-mixin}.

\Defclass {generic-function-optional-parameter}

This class represents an optional parameter of the form that is
allowed in an ordinary lambda list, i.e., a parameter that can only
have a name, but that name can optionally be the element of a
singleton list, which is why it is distinct from a parameter of type
\texttt{simple-variable}.

This class is a subclass of the class \texttt{parameter}.

\Defclass {specialized-required-parameter}

This class represents the kind of required parameter that can appear
in a specialized lambda list, i.e. a lambda list in a
\texttt{defmethod} form.  A parameter of this type may optionally have
a \emph{specializer} in the form of a class name or an \texttt{eql}
specializer associated with it.

This class is a subclass of the class \texttt{parameter}.

\Defclass {destructuring-parameter}

This class represents all parameters that can be either simple
variables or a pattern in the form of a (possibly dotted) list, or
even a destructuring lambda list.

This class is a subclass of the class \texttt{grammar-symbol}.

\subsection{Classes for lambda-list keywords}

\Defclass {lambda-list-keyword}

This class is the root class of all classes representing lambda-list
keywords.

This class is a subclass of the class \texttt{grammar-symbol}.

\Defgeneric {name} {lambda-list-keyword}

This generic function returns the concrete syntax tree corresponding
to the lambda-list keyword in the original lambda list.  It is
applicable to every instantiable subclass of the class
\texttt{lambda-list-keyword}.

\Defclass {keyword-optional}

This class represents the lambda-list keyword \texttt{\&optional}.
Unless client code has defined some additional lambda-list keyword
that is used in the same way as \texttt{\&optional}, an instance of
this class always represents the lambda-list keyword
\texttt{\&optional}.

This class is a subclass of the class \texttt{lambda-list-keyword}.

\Defclass {keyword-rest}

This class represents the lambda-list keyword \texttt{\&rest}.
Unless client code has defined some additional lambda-list keyword
that is used in the same way as \texttt{\&rest}, an instance of
this class always represents the lambda-list keyword
\texttt{\&rest}.

This class is a subclass of the class \texttt{lambda-list-keyword}.

\Defclass {keyword-body}

This class represents the lambda-list keyword \texttt{\&body}.
Unless client code has defined some additional lambda-list keyword
that is used in the same way as \texttt{\&body}, an instance of
this class always represents the lambda-list keyword
\texttt{\&body}.

This class is a subclass of the class \texttt{lambda-list-keyword}.

\Defclass {keyword-key}

This class represents the lambda-list keyword \texttt{\&key}.
Unless client code has defined some additional lambda-list keyword
that is used in the same way as \texttt{\&key}, an instance of
this class always represents the lambda-list keyword
\texttt{\&key}.

This class is a subclass of the class \texttt{lambda-list-keyword}.

\Defclass {keyword-allow-other-keys}

This class represents the lambda-list keyword \texttt{\&allow-other-keys}.
Unless client code has defined some additional lambda-list keyword
that is used in the same way as \texttt{\&allow-other-keys}, an instance of
this class always represents the lambda-list keyword
\texttt{\&allow-other-keys}.

This class is a subclass of the class \texttt{lambda-list-keyword}.

\Defclass {keyword-aux}

This class represents the lambda-list keyword \texttt{\&aux}.
Unless client code has defined some additional lambda-list keyword
that is used in the same way as \texttt{\&aux}, an instance of
this class always represents the lambda-list keyword
\texttt{\&aux}.

This class is a subclass of the class \texttt{lambda-list-keyword}.

\Defclass {keyword-environment}

This class represents the lambda-list keyword \texttt{\&environment}.
Unless client code has defined some additional lambda-list keyword
that is used in the same way as \texttt{\&environment}, an instance of
this class always represents the lambda-list keyword
\texttt{\&environment}.

This class is a subclass of the class \texttt{lambda-list-keyword}.

\Defclass {keyword-whole}

This class represents the lambda-list keyword \texttt{\&whole}.
Unless client code has defined some additional lambda-list keyword
that is used in the same way as \texttt{\&whole}, an instance of
this class always represents the lambda-list keyword
\texttt{\&whole}.

This class is a subclass of the class \texttt{lambda-list-keyword}.

\subsection{Classes for entire lambda lists}

Notice that there is no class for the boa lambda list, since it is
syntactically equivalent to an ordinary lambda list.  Similarly, there
is no \texttt{deftype} lambda list because it is syntactically
equivalent to a macro lambda list.

\Defclass {lambda-list-type}

This class is a subclass of the class \texttt{grammar-symbol}.

\Defgeneric {children} {lambda-list-type}

This generic function returns a (possibly empty) list of instances of
(some subclass of) the class \texttt{parameter-group} as described in
\refSec{sec-parsing-lambda-lists-parameter-groups}.

\Defclass {ordinary-lambda-list}

Applying the function \texttt{children} to an instance of this class
returns a list with the following elements (in that order):

\begin{enumerate}
\item A mandatory instance of the grammar-symbol class named\\
  \texttt{ordinary-required-parameter-group}.
\item An optional instance of the grammar-symbol class named\\
  \texttt{ordinary-optional-parameter-group}.
\item An optional instance of the grammar-symbol class named\\
\texttt{ordinary-rest-parameter-group}.
\item An optional instance of the grammar-symbol class named\\
\texttt{ordinary-key-parameter-group}.
\item An optional instance of the grammar-symbol class named\\
\texttt{aux-parameter-group}.
\end{enumerate}

This class is a subclass of the class \texttt{lambda-list-type}.

\Defclass {generic-function-lambda-list}

Applying the function \texttt{children} to an instance of this class
returns a list with the following elements (in that order):

\begin{enumerate}
\item A mandatory instance of the grammar-symbol class named\\
  \texttt{ordinary-required-parameter-group}.
\item An optional instance of the grammar-symbol class named\\
\texttt{generic-function-optional-parameter-group},
\item An optional instance of the grammar-symbol class named\\
\texttt{ordinary-rest-parameter-group}.
\item An optional instance of the grammar-symbol class named\\
\texttt{generic-function-key-parameter-group}.
\item An optional instance of the grammar-symbol class named\\
\texttt{aux-parameter-group}.
\end{enumerate}

This class is a subclass of the class \texttt{lambda-list-type}.

\Defclass {specialized-lambda-list}

Applying the function \texttt{children} to an instance of this class
returns a list with the following elements (in that order):

\begin{enumerate}
\item A mandatory instance of the grammar-symbol class named\\
  \texttt{specialized-required-parameter-group}.
\item An optional instance of the grammar-symbol class named\\
  \texttt{ordinary-optional-parameter-group}.
\item An optional instance of the grammar-symbol class named\\
\texttt{ordinary-rest-parameter-group}.
\item An optional instance of the grammar-symbol class named\\
\texttt{ordinary-key-parameter-group}.
\item An optional instance of the grammar-symbol class named\\
\texttt{aux-parameter-group}.
\end{enumerate}

This class is a subclass of the class \texttt{lambda-list-type}.

\Defclass {defsetf-lambda-list}

Applying the function \texttt{children} to an instance of this class
returns a list with the following elements (in that order):

\begin{enumerate}
\item A mandatory instance of the grammar-symbol class named\\
  \texttt{ordinary-required-parameter-group}.
\item An optional instance of the grammar-symbol class named\\
  \texttt{ordinary-optional-parameter-group}.
\item An optional instance of the grammar-symbol class named\\
\texttt{ordinary-rest-parameter-group}.
\item An optional instance of the grammar-symbol class named\\
\texttt{ordinary-key-parameter-group}.
\item An optional instance of the grammar-symbol class named\\
\texttt{environment-parameter-group}.
\end{enumerate}

This class is a subclass of the class \texttt{lambda-list-type}.

\Defclass {define-modify-macro-lambda-list}

Applying the function \texttt{children} to an instance of this class
returns a list with the following elements (in that order):

\begin{enumerate}
\item A mandatory instance of the grammar-symbol class named\\
  \texttt{ordinary-required-parameter-group}.
\item An optional instance of the grammar-symbol class named\\
  \texttt{ordinary-optional-parameter-group}.
\item An optional instance of the grammar-symbol class named\\
\texttt{ordinary-rest-parameter-group}.
\end{enumerate}

This class is a subclass of the class \texttt{lambda-list-type}.

\Defclass {define-method-combination-lambda-list}

Applying the function \texttt{children} to an instance of this class
returns a list with the following elements (in that order):

\begin{enumerate}
\item An optional instance of the grammar-symbol class named\\
  \texttt{whole-parameter-group}.
\item A mandatory instance of the grammar-symbol class named\\
  \texttt{ordinary-required-parameter-group}.
\item An optional instance of the grammar-symbol class named\\
  \texttt{ordinary-optional-parameter-group}.
\item An optional instance of the grammar-symbol class named\\
\texttt{ordinary-rest-parameter-group}.
\item An optional instance of the grammar-symbol class named\\
\texttt{ordinary-key-parameter-group}.
\item An optional instance of the grammar-symbol class named\\
\texttt{aux-parameter-group}.
\end{enumerate}

This class is a subclass of the class \texttt{lambda-list-type}.

\Defclass {macro-lambda-list}

This class is a subclass of the class \texttt{lambda-list-type}.

\Defclass {destructuring-lambda-list}

This class is a subclass of the class \texttt{lambda-list-type}.

\Defclass {target}

This class is a subclass of the class \texttt{grammar-symbol}.

\section{Variables}

\subsection{Parameter groups}

\Defvar {*ordinary-required-parameter-group*}

This variable defines a grammar rule for the grammar-symbol class
named \texttt{ordinary-required-parameter-group}.  This rule defines
an ordinary required parameter group as a (possibly empty) sequence of
instances of the class \texttt{simple-variable}.

\Defvar {*ordinary-optional-parameter-group*}

This variable defines a grammar rule for the grammar-symbol class
named \texttt{ordinary-optional-parameter-group}.  This rule defines
an ordinary optional parameter group as the lambda list keyword
\texttt{\&optional}, followed by a (possibly empty) sequence of
instances of the class \texttt{ordinary-optional-parameter}.

\Defvar {*ordinary-rest-parameter-group*}

This variable defines a grammar rule for the grammar-symbol class
named \texttt{ordinary-rest-parameter-group}.  This rule defines an
ordinary rest parameter group as the lambda list keyword
\texttt{\&rest}, followed by an instances of the class
\texttt{simple-variable}.

\Defvar {*ordinary-key-parameter-group*}

This variable defines a grammar rule for the grammar-symbol class
named \texttt{ordinary-key-parameter-group}.  This rule defines
an ordinary key parameter group as the lambda list keyword
\texttt{\&key}, followed by a (possibly empty) sequence of
instances of the class \texttt{ordinary-key-parameter}, optionally
followed by the lambda-list keyword \texttt{\&allow-other-keys}.

\Defvar {*aux-parameter-group*}

This variable defines a grammar rule for the grammar-symbol class
named \texttt{aux-parameter-group}.  This rule defines an aux
parameter group as the lambda list keyword \texttt{\&aux}, followed by
a (possibly empty) sequence of instances of the class
\texttt{aux-parameter}.

\Defvar {*generic-function-optional-parameter-group*}

This variable defines a grammar rule for the grammar-symbol class
named \texttt{generic-function-optional-parameter-group}.  This rule
defines a generic-function optional parameter group as the lambda list
keyword \texttt{\&optional}, followed by a (possibly empty) sequence
of instances of the class named\\
\texttt{generic-function-optional-parameter}.

\Defvar {*generic-function-key-parameter-group*}

This variable defines a grammar rule for the grammar-symbol class
named \texttt{generic-function-key-parameter-group}.  This rule
defines an generic-function key parameter group as the lambda list
keyword \texttt{\&key}, followed by a (possibly empty) sequence of
instances of the class \texttt{generic-function-key-parameter},
optionally followed by the lambda-list keyword
\texttt{\&allow-other-keys}.

\Defvar {*specialized-required-parameter-group*}

This variable defines a grammar rule for the grammar-symbol class
named \texttt{specialized-required-parameter-group}.  This rule
defines an specialized required parameter group as a (possibly empty)
sequence of instances of the class
\texttt{specialized-required-parameter}.

\Defvar {*environment-parameter-group*}

This variable defines a grammar rule for the grammar-symbol class
named \texttt{environment-parameter-group}.  This rule defines an
environment parameter group as the lambda list keyword
\texttt{\&environment}, followed by an instances of the class
\texttt{simple-variable}.

\Defvar {*whole-parameter-group*}

This variable defines a grammar rule for the grammar-symbol class
named \texttt{whole-parameter-group}.  This rule defines a whole
parameter group as the lambda list keyword \texttt{\&whole}, followed
by an instances of the class \texttt{simple-variable}.

\Defvar {*destructuring-required-parameter-group*}

This variable defines a grammar rule for the grammar-symbol class
named \texttt{destructuring-required-parameter-group}.  This rule
defines an destructuring required parameter group as a (possibly
empty) sequence of instances of the class
\texttt{destructuring-parameter}.

\Defvar {*destructuring-rest-parameter-group*}

This variable defines a grammar rule for the grammar-symbol class
named \texttt{destructuring-rest-parameter-group}.  This rule defines
a destructuring rest parameter group as the lambda list keyword
\texttt{\&rest}, followed by an instances of the class
\texttt{destructuring-parameter}.

\subsection{Lambda-list types}

\Defvar {*ordinary-lambda-list*}

This variable defines a grammar rule for the grammar-symbol class
named \texttt{ordinary-lambda-list}.  It has the following definition:

\begin{verbatim}
(defparameter *ordinary-lambda-list*
  '((ordinary-lambda-list <-
     ordinary-required-parameter-group
     (? ordinary-optional-parameter-group)
     (? ordinary-rest-parameter-group)
     (? ordinary-key-parameter-group)
     (? aux-parameter-group))))
\end{verbatim}

\Defvar {*generic-function-lambda-list*}

This variable defines a grammar rule for the grammar-symbol class
named \texttt{generic-function-lambda-list}.  It has the following
definition:

\begin{verbatim}
(defparameter *generic-function-lambda-list*
  '((generic-function-lambda-list <-
     ordinary-required-parameter-group
     (? generic-function-optional-parameter-group)
     (? ordinary-rest-parameter-group)
     (? generic-function-key-parameter-group))))
\end{verbatim}

\Defvar {*specialized-lambda-list*}

This variable defines a grammar rule for the grammar-symbol class
named \texttt{specialized-lambda-list}.  It has the following
definition:

\begin{verbatim}
(defparameter *specialized-lambda-list*
  '((specialized-lambda-list <-
     specialized-required-parameter-group
     (? ordinary-optional-parameter-group)
     (? ordinary-rest-parameter-group)
     (? ordinary-key-parameter-group)
     (? aux-parameter-group))))
\end{verbatim}

\Defvar {*defsetf-lambda-list*}

This variable defines a grammar rule for the grammar-symbol class
named \texttt{defsetf-lambda-list}.  It has the following
definition:

\begin{verbatim}
(defparameter *defsetf-lambda-list*
  '((defsetf-lambda-list <-
     ordinary-required-parameter-group
     (? ordinary-optional-parameter-group)
     (? ordinary-rest-parameter-group)
     (? ordinary-key-parameter-group)
     (? environment-parameter-group))))
\end{verbatim}

\Defvar {*define-modify-macro-lambda-list*}

This variable defines a grammar rule for the grammar-symbol class
named \texttt{define-modify-macro-lambda-list}.  It has the following
definition:

\begin{verbatim}
(defparameter *define-modify-macro-lambda-list*
  '((define-modify-macro-lambda-list <-
     ordinary-required-parameter-group
     (? ordinary-optional-parameter-group)
     (? ordinary-rest-parameter-group))))
\end{verbatim}

\Defvar {*define-method-combination-lambda-list*}

This variable defines a grammar rule for the grammar-symbol class
named \texttt{define-method-combination-lambda-list}.  It has the
following definition:

\begin{verbatim}
(defparameter *define-method-combination-lambda-list*
  '((define-method-combination-lambda-list <-
     (? whole-parameter-group)
     ordinary-required-parameter-group
     (? ordinary-optional-parameter-group)
     (? ordinary-rest-parameter-group)
     (? ordinary-key-parameter-group)
     (? aux-parameter-group))))
\end{verbatim}

\Defvar {*destructuring-lambda-list*}

This variable defines a grammar rule for the grammar-symbol class
named \texttt{destructuring-lambda-list}.

This rule defines a destructuring lambda list as sequence of the
following items:

\begin{enumerate}
\item An optional instance of the grammar-symbol class named\\
  \texttt{whole-parameter-group}.
\item A mandatory instance of the grammar-symbol class named\\
  \texttt{destructuring-required-parameter-group}.
\item An optional instance of the grammar-symbol class named\\
  \texttt{ordinary-optional-parameter-group}.
\item An optional instance of the grammar-symbol class named\\
\texttt{destructuring-rest-parameter-group}.
\item An optional instance of the grammar-symbol class named\\
\texttt{ordinary-key-parameter-group}.
\item An optional instance of the grammar-symbol class named\\
\texttt{aux-parameter-group}.
\end{enumerate}

\Defvar {*macro-lambda-list*}

This variable defines a grammar rule for the grammar-symbol class
named \texttt{macro-lambda-list}.

This rule defines a macro lambda list as sequence of the
following items:

\begin{enumerate}
\item An optional instance of the grammar-symbol class named\\
  \texttt{whole-parameter-group}.
\item An optional instance of the grammar-symbol class named\\
  \texttt{environment-parameter-group}.
\item A mandatory instance of the grammar-symbol class named\\
  \texttt{destructuring-required-parameter-group}.
\item An optional instance of the grammar-symbol class named\\
  \texttt{environment-parameter-group}.
\item An optional instance of the grammar-symbol class named\\
  \texttt{ordinary-optional-parameter-group}.
\item An optional instance of the grammar-symbol class named\\
  \texttt{environment-parameter-group}.
\item An optional instance of the grammar-symbol class named\\
\texttt{destructuring-rest-parameter-group}.
\item An optional instance of the grammar-symbol class named\\
  \texttt{environment-parameter-group}.
\item An optional instance of the grammar-symbol class named\\
\texttt{ordinary-key-parameter-group}.
\item An optional instance of the grammar-symbol class named\\
  \texttt{environment-parameter-group}.
\item An optional instance of the grammar-symbol class named\\
\texttt{aux-parameter-group}.
\item An optional instance of the grammar-symbol class named\\
  \texttt{environment-parameter-group}.
\end{enumerate}

Notice that this definition allows for there to be several occurrences
of the grammar symbol \texttt{environment-parameter-group}, whereas
the \commonlisp{} standard allows for at most one such occurrence.
The top-level parser for this type of lambda list checks that at most
one such occurrence is present after parsing is complete.

\subsection{Full grammars}

In order to have a grammar that is possible to use for parsing a
lambda list, in addition to the rules for all the lambda list types
and its components, a special rule for the grammar symbol
\texttt{target} is required.  A different rule for this grammar symbol
is added to the grammar, according to the particular lambda-list type
that is desired.

\Defvar {*standard-grammar*}

This variable contains all the standard grammar rules, excluding a
rule for the grammar symbol \texttt{target}.

\Defvar {*ordinary-lambda-list-grammar*}

This variable contains all the standard grammar rules, plus a rule
with \texttt{target} on its left-hand side and
\texttt{ordinary-lambda-list} as its right-hand side.

\Defvar {*generic-function-lambda-list-grammar*}

This variable contains all the standard grammar rules, plus a rule
with \texttt{target} on its left-hand side and
\texttt{generic-function-lambda-list} as its right-hand side.

\Defvar {*specialized-lambda-list-grammar*}

This variable contains all the standard grammar rules, plus a rule
with \texttt{target} on its left-hand side and
\texttt{specialized-lambda-list} as its right-hand side.

\Defvar {*defsetf-lambda-list-grammar*}

This variable contains all the standard grammar rules, plus a rule
with \texttt{target} on its left-hand side and
\texttt{defsetf-lambda-list} as its right-hand side.

\Defvar {*define-modify-macro-lambda-list-grammar*}

This variable contains all the standard grammar rules, plus a rule
with \texttt{target} on its left-hand side and
\texttt{define-modify-macro-lambda-list} as its right-hand side.

\Defvar {*define-method-combination-lambda-list-grammar*}

This variable contains all the standard grammar rules, plus a rule
with \texttt{target} on its left-hand side and
\texttt{define-method-combination-lambda-list} as its right-hand side.

\Defvar {*destructuring-lambda-list-grammar*}

This variable contains all the standard grammar rules, plus a rule
with \texttt{target} on its left-hand side and
\texttt{destructuring-lambda-list} as its right-hand side.

\Defvar {*macro-lambda-list-grammar*}

This variable contains all the standard grammar rules, plus a rule
with \texttt{target} on its left-hand side and
\texttt{macro-lambda-list} as its right-hand side.

\section{Parsers for standard lambda lists}

\Defun {parse-ordinary-lambda-list}\\
{client lambda-list \key (error-p t)}

\Defun {parse-generic-function-lambda-list}\\
{client lambda-list \key (error-p t)}

\Defun {parse-specialized-lambda-list}\\
{client lambda-list \key (error-p t)}

\Defun {parse-defsetf-lambda-list}\\
{client lambda-list \key (error-p t)}

\Defun {parse-define-modify-macro-lambda-list}\\
{client lambda-list \key (error-p t)}

\Defun {parse-define-method-combination-lambda-list}\\
{client lambda-list \key (error-p t)}

\Defun {parse-destructuring-lambda-list}\\
{client lambda-list \key (error-p t)}

\Defun {parse-macro-lambda-list}\\
{client lambda-list \key (error-p t)}
