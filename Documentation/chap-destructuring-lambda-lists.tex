\chapter{Destructuring lambda lists}
\label{chap-destructuring-lambda-lists}

When applied to lambda lists, the term ``destructuring'' means to
match its parameters against an argument list, and to generate a set
of nested \texttt{let} bindings.  A binding will bind a parameter of
the lambda list to its corresponding value in the argument list, or it
will bind some temporary variable.  The argument list is not known at
the time of the destructuring, so the \emph{form} of each binding will
consist of calls to destructuring functions such as \texttt{car} and
\texttt{cdr}, starting with a variable that holds the entire argument
list as its value.

This kind of destructuring is used at macro-expansion time when
certain macros are expanded.  In particular \texttt{defmacro} and
\texttt{define-compiler-macro}.  The result of the destructuring is a
\texttt{lambda} expression for the \emph{macro function}.  This lambda
expression is then compiled to create the final macro function.

Every function defined here wraps a \textit{body} form in some
\texttt{let} bindings.  These \texttt{let} bindings are determined by
the parameters of a lambda list.  Each function handles a different
part of the lambda list.  The \textit{client} parameter is some object
representing the client.  It is used among other things to determine
which condition class to use when a a condition needs to be signaled.
The \textit{argument-variable} parameter (abbreviated \textit{av} is a
symbol that, when the resulting macro function is executed on some
compound form corresponding to a macro call, will hold the remaining
part of the arguments of that macro call yet to be processed.

Some functions have an argument called \textit{tail-variable}
(abbreviated \textit{tv}), which is also a symbol that is going to be
used in subsequent destructuring functions for the same purpose as
\textit{argument-variable}.  Such a function is responsible for
creating an innermost \texttt{let} form that binds the
\textit{tail-variable} symbol to the part of the argument list that
remains after the function has done its processing.  Some functions do
not need such a variable, because they do not consume any arguments,
so the remaining argument list is the same as the initial one.

\Defgeneric {destructure-lambda-list} {client lambda-list av tv body}

Given an entire lambda list, which can be a macro lambda list or a
destructuring lambda list, Wrap \textit{body} in a bunch of nested
\texttt{let} bindings according to the parameters of the lambda list.

\Defgeneric {destructure-aux-parameter} {client parameter body}

Wrap \textit{body} in a \texttt{let} form corresponding to a single
\texttt{aux} parameter.  Since \texttt{aux} parameters are independent
of the macro-call arguments, there is no need for an
\texttt{argument-variable}.  The \texttt{aux} parameter itself
provides all the information required to determine the \texttt{let}
binding.

\Defgeneric {destructure-aux-parameters} {client parameters body}

Wrap \textit{body} in nested \texttt{let} forms, each corresponding to
a single \texttt{aux} parameter in the list of \texttt{aux} parameters
\textit{parameters}.  Since \texttt{aux} parameters are independent of
the macro-call argument, there is no need for an
\textit{argument-variable}.  Each \texttt{aux} parameter in
\textit{parameters} itself provides all the information required to
determine the \texttt{let} binding.

\Defgeneric {destructure-key-parameter} {client parameter av body}

Wrap \textit{body} in a \texttt{let} form corresponding to a single
\texttt{key} parameter.

\Defgeneric {destructure-key-parameters} {client parameters av body}

Wrap \textit{body} in nested \texttt{let} forms, each corresponding to
a single \texttt{key} parameter in a list of such \texttt{key}
parameters.  Since \texttt{key} parameters do not consume any
arguments, the list of arguments is the same before and after the
\texttt{key} parameters have been processed.  As a consequence, we do
not need a \textit{tail-variable} for \texttt{key} parameters.

\Defgeneric {destructure-rest-parameter} {client parameter av body}

Wrap \textit{body} in a \texttt{let} form corresponding to a
\texttt{rest} parameter.  Since \texttt{rest} parameters do not
consume any arguments, the list of arguments is the same before and
after the \texttt{rest} parameter has been processed.  As a
consequence, we do not need a \textit{tail-variable} for \texttt{rest}
parameters.

\Defgeneric {destructure-optional-parameter} {client parameter av body}

Wrap \textit{body} in a \texttt{let} form corresponding to a single
\texttt{optional} parameter.

\Defgeneric {destructure-optional-parameters} {client parameters av tv body}

Wrap \textit{body} in nested \texttt{let} forms, each corresponding to
a single \texttt{optional} parameter in a list of such
\texttt{optional} parameters.  Since every \texttt{optional} parameter
\textit{does} consume an argument, this function does take a
\texttt{tail-variable} argument as described above.

\Defgeneric {destructure-required-parameter} {client parameter av body}

Wrap \textit{body} in one or more \texttt{let} forms corresponding to
a single required parameter, depending on whether the required
parameter is a simple variable or a destructuring lambda list.

\Defgeneric {destructure-required-parameters} {client parameters av tv body}

Wrap \textit{body} in nested \texttt{let} forms, corresponding to the
list of required parameters in the list of required parameters
\textit{parameters}.  Since every required parameter \textit{does}
consume an argument, this function does take a \texttt{tail-variable}
argument as described above.

\Defgeneric {destructure-parameter-group} {client group av tv body}

